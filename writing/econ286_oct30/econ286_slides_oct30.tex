\documentclass[notes,11pt, aspectratio=169]{beamer}

\usepackage{pgfpages}
\usepackage[
backend=biber,
style=authoryear,
]{biblatex}

\DeclareFieldFormat{parens}{\mkbibparens{#1}}
\renewbibmacro*{cite}{\printnames{labelname}\setunit{\addspace}\printfield[parens]{year}}

% These slides also contain speaker notes. You can print just the slides,
% just the notes, or both, depending on the setting below. Comment out the want
% you want.
 \setbeameroption{hide notes} % Only slide
%\setbeameroption{show only notes} % Only notes
% \setbeameroption{show notes on second screen=right} % Both

\usepackage{booktabs}
\usepackage{siunitx}
\newcolumntype{d}{S[
    input-open-uncertainty=,
    input-close-uncertainty=,
    parse-numbers = false,
    table-align-text-pre=false,
    table-align-text-post=false
 ]}

\usepackage{helvet}
\usepackage[default]{lato}
\usepackage{bm}
\usepackage{array}
\usepackage{adjustbox}
\newcommand{\tabnotes}[2]{\bottomrule \multicolumn{#1}{@{}p{0.7\linewidth}@{}}{\footnotesize #2 }\end{tabular}\end{table}}
\usepackage{tikz}
\usetikzlibrary{positioning,shapes,arrows,calc,decorations.pathreplacing,fit}
\usepackage{verbatim}
\setbeamertemplate{note page}{\pagecolor{yellow!5}\insertnote}

\usepackage{changepage}
\usepackage{appendixnumberbeamer}

\newcommand{\beginbackup}{
   \newcounter{framenumbervorappendix}
   \setcounter{framenumbervorappendix}{\value{framenumber}}
   \setbeamertemplate{footline}
   {
     \leavevmode%
     \hline
     box{%
       \begin{beamercolorbox}[wd=\paperwidth,ht=2.25ex,dp=1ex,right]{footlinecolor}%
%         \insertframenumber  \hspace*{2ex}
       \end{beamercolorbox}}%
     \vskip0pt%
   }
 }
\newcommand{\backupend}{
   \addtocounter{framenumbervorappendix}{-\value{framenumber}}
   \addtocounter{framenumber}{\value{framenumbervorappendix}}
}

\PassOptionsToPackage{table}{xcolor}
\usepackage{comment}
\usepackage{graphicx}
\usepackage[space]{grffile}
\usepackage{booktabs}

% These are my colors -- there are many like them, but these ones are mine.
\definecolor{blue}{RGB}{64, 114, 190}
\definecolor{red}{RGB}{213, 60, 50}
\definecolor{yellow}{RGB}{240,228,66}
\definecolor{green}{RGB}{0,158,115}
\definecolor{chiffon}{RGB}{235,179,121}
\definecolor{peach}{RGB}{228,167,125}
\definecolor{lilac}{RGB}{180,125,228}

\hypersetup{
  colorlinks=false,
  linkbordercolor = {white},
  linkcolor = {blue}
}

%% I use a beige off white for my background
\definecolor{MyBackground}{RGB}{255,253,218}
\setbeamercovered{transparent=50}

%% Uncomment this if you want to change the background color to something else
%\setbeamercolor{background canvas}{bg=MyBackground}

%% Change the bg color to adjust your transition slide background color!
\newenvironment{transitionframe}{
  \setbeamercolor{background canvas}{bg=chiffon}
  \begin{frame}}
  {
    \end{frame}}

\setbeamercolor{frametitle}{fg=blue}
\setbeamercolor{title}{fg=black}
\setbeamertemplate{footline}[frame number]
\setbeamertemplate{navigation symbols}{}
\setbeamertemplate{itemize items}{-}
\setbeamercolor{itemize item}{fg=blue}
\setbeamercolor{itemize subitem}{fg=blue}
\setbeamercolor{enumerate item}{fg=blue}
\setbeamercolor{enumerate subitem}{fg=blue}
\setbeamercolor{button}{bg=MyBackground,fg=blue,}
\addbibresource{../bib/references.bib}

% If you like road maps, rather than having clutter at the top, have a roadmap show up at the end of each section
% (and after your introduction)
% Uncomment this is if you want the roadmap!
\AtBeginSection[]
{
   \begin{frame}
       \frametitle{Roadmap of Talk}
       \tableofcontents[currentsection]
   \end{frame}
}
\setbeamercolor{section in toc}{fg=blue}
\setbeamercolor{subsection in toc}{fg=red}
\setbeamersize{text margin left=1em,text margin right=1em}

\newenvironment{wideitemize}{\itemize\addtolength{\itemsep}{10pt}}{\enditemize}
\usepackage{environ}
\NewEnviron{videoframe}[1]{
  \begin{frame}
    \vspace{-8pt}
    \begin{columns}[onlytextwidth, T] % align columns
      \begin{column}{.58\textwidth}
        \begin{minipage}[t][\textheight][t]
          {\dimexpr\textwidth}
          \vspace{8pt}
          \hspace{4pt} {\Large \sc \textcolor{blue}{#1}}
          \vspace{8pt}

          \BODY
        \end{minipage}
      \end{column}%
      \hfill%
      \begin{column}{.42\textwidth}
        \colorbox{green!20}{\begin{minipage}[t][1.2\textheight][t]
            {\dimexpr\textwidth}
            Face goes here
          \end{minipage}}
      \end{column}%
    \end{columns}
  \end{frame}
}

\usepackage{graphicx}
\usepackage{bbm}
\usepackage{multicol}
\usepackage{mathtools}
\usepackage{amsmath}
\usepackage{amssymb}
\usepackage{xcolor}
\usepackage{amsopn}
\usepackage{booktabs}
\usepackage{tikz}
\usepackage{graphicx}
\usepackage{multirow}
\usepackage{pdflscape}
\usepackage{lscape}
\usepackage{tikz}
\usepackage{amsfonts}
\usepackage{caption}
\usepackage{xcolor}
\usepackage{subcaption}
\usepackage{colortbl}
\usepackage{cancel}
\usepackage{soul}
\usepackage{tcolorbox}

\makeatletter
\let\HL\hl
\renewcommand\hl{%
  \let\set@color\beamerorig@set@color
  \let\reset@color\beamerorig@reset@color
  \HL}
\makeatother
\usetikzlibrary{patterns}
\newcommand{\mathcolorbox}[2]{\colorbox{#1}{$\displaystyle #2$}}
\newtheorem{hyp}{Assumption}
% Tikz settings optimized for causal graphs.
% Just copy-paste this part
\usetikzlibrary{shapes,decorations,arrows,calc,arrows.meta,fit,positioning}
\newcommand{\tikzmark}[1]{\tikz[overlay,remember picture] \node (#1) {};}

\tikzset{
    -Latex,auto,node distance =1 cm and 1 cm,semithick,
    state/.style ={ellipse, draw, minimum width = 0.7 cm},
    point/.style = {circle, draw, inner sep=0.04cm,fill,node contents={}},
    bidirected/.style={Latex-Latex,dashed},
    el/.style = {inner sep=2pt, align=left, sloped}
}

\tikzstyle{startstop} = [rectangle, rounded corners,
minimum width=3cm,
minimum height=1cm,
text centered,
draw=black,
fill=red!30]

\tikzstyle{io} = [trapezium,
trapezium stretches=true, % A later addition
trapezium left angle=70,
trapezium right angle=110,
minimum width=3cm,
minimum height=1cm, text centered,
draw=black, fill=blue!30]

\tikzstyle{process} = [rectangle,
minimum width=3cm,
minimum height=1cm,
text centered,
text width=3cm,
draw=black,
fill=orange!30]

\tikzstyle{decision} = [diamond,
minimum width=3cm,
minimum height=1cm,
text centered,
draw=black,
fill=green!30]
\tikzstyle{arrow} = [thick,->,>=stealth]

\usetikzlibrary{matrix}
\newcommand\Wider[2][3em]{%
\makebox[\linewidth][c]{%
  \begin{minipage}{\dimexpr\textwidth+#1\relax}
  \raggedright#2
  \end{minipage}%
  }%
}

%%%% Setup headers: thanks to Sagar Saxena for this code
\setbeamercolor{footerColor}{fg=peach,bg=white}
\setbeamertemplate{headline}{%
  \leavevmode%
  \hbox{%
    \begin{beamercolorbox}[wd=\paperwidth,ht=2.5ex,dp=1.125ex,sep=0pt,colsep=0pt]{footerColor}%
      \ifx\insertsection\empty
        % it is empty
      \else
        \vspace*{-\fboxsep}\colorbox{peach}{\color{white}\textbf{\insertsectionhead}}
      \fi
      % Check if subsection is not empty and display it
      \ifx\insertsubsectionhead\empty
        % Subsection is empty
      \else
        \hspace{.1cm}% Space between section and subsection
        \textbf{\insertsubsectionhead}%
      \fi
    \end{beamercolorbox}%
  }%
}

\setbeamertemplate{footline}[frame number]
\setbeamertemplate{navigation symbols}

\DeclareMathOperator*{\argmin}{arg\!\min}
\DeclareMathOperator*{\argmax}{arg\!\max}

\definecolor {processblue}{cmyk}{0.96,0,0,0}
\newcolumntype{t}{>{\columncolor{red!20}}c}

%----------------------------------------------------------------------------------------
%    TITLE PAGE
%----------------------------------------------------------------------------------------

\title[]{How the Dawn of Public Higher Education (1900-1940) Shaped Access and Work}
\author{Collin Wardius} % Your name
\institute{
  Department of Economics, UC San Diego
  \newline
  Approved by
}
\date[]{October 30, 2025}

\begin{document}

\begin{frame}
  \titlepage
\end{frame}

\begin{frame}{Higher education in the US experienced its first major transformation in the early 1900s}
  \begin{wideitemize}
    \item Many more students enrolled
    \item Public universities began to dominate in terms of enrollment
  \end{wideitemize}
\end{frame}


\begin{frame}{Questions}
\begin{wideitemize}
    \item \textbf{How did the founding of public colleges change access to college?}
    \item How did the founding of public colleges change the labor force of local economies?
\end{wideitemize}
\end{frame}

\begin{frame}{Preview of identification approach}
  \begin{wideitemize}
    \item \textbf{Identifying variation}: quasi-random founding date of a university
    \item Some people are lucky as they are born just late enough to access a new university
    \item Some people are unlucky as they are born too early to access a new university
  \end{wideitemize}
\end{frame}

\begin{frame}{Literature}
  \begin{itemize}
    \item \textbf{History of US higher education (1900-1940)}: \cite{goldinAmericasGraduationHigh1998}, \cite{goldinOriginsStateLevelDifferences1998}, \cite{goldinHumanCapitalCenturyAmerican2001}
    \begin{itemize}
      \item[\textcolor{blue}{$\rightarrow$}] \textit{My contribution:} Quantify the causal effect of university expansion on education access
    \end{itemize}

    \item \textbf{Effects of university building in non-US countries}: \cite{dufloSchoolingLaborMarket2001}, \cite{nimier-davidLocalHumanCapital2023}
    \begin{itemize}
      \item[\textcolor{blue}{$\rightarrow$}] \textit{My contribution:} US university foundings and variation in public vs private control
    \end{itemize}

    \item \textbf{How proximity to college affects attainment}: \cite{cardUsingGeographicVariation1993}, \cite{actonDistanceDegreesHow2025}
    \begin{itemize}
      \item[\textcolor{blue}{$\rightarrow$}] \textit{My contribution:} Examine extensive margin of college access via new university foundings
    \end{itemize}
  \end{itemize}
\end{frame}



\begin{frame}{BA Completion: 1900 vs 1936 Birth Cohorts}
    \begin{figure}
        \centering
        \includegraphics[width=0.8\textwidth,height=0.7\textheight,keepaspectratio]{"/Users/cjwardius/Library/CloudStorage/OneDrive-UCSanDiego/demo of education/output/figures/ba_completion_1900_vs_1936.png"}
        \caption{BA Completion: 1900 vs 1936 Birth Cohorts}
    \end{figure}
\end{frame}

\begin{frame}{College Founding Years by Region}
    \begin{figure}
        \centering
        \includegraphics[width=0.8\textwidth,height=0.7\textheight,keepaspectratio]{"/Users/cjwardius/Library/CloudStorage/OneDrive-UCSanDiego/demo of education/output/figures/founding_years_cdf_regional.png"}
        \caption{Regional Distribution of College Founding Years}
    \end{figure}
\end{frame}

\begin{frame}{College Founding Years by Control}
    \begin{figure}
        \centering
        \includegraphics[width=0.8\textwidth,height=0.7\textheight,keepaspectratio]{"/Users/cjwardius/Library/CloudStorage/OneDrive-UCSanDiego/demo of education/output/figures/founding_years_cdf_control_type_1800_plus.png"}
        \caption{Regional Distribution of College Founding Years (1800+)}
    \end{figure}
\end{frame}


\begin{frame}{Estimating the effect of a university founding on college attainment}
  Cross sectional regression, identifying variation is at the cohort-by-county level.
  \begin{equation}
    y_{ick} = \alpha_c + \lambda_k + \beta \text{New college}_{ck}\times \lambda_k +\xi \bm{X}_{ick} + \epsilon_{ick}
  \end{equation}
  \begin{wideitemize}
    \item $c$: county, $k$: age cohort, $i$: person
    \item $\text{New college}_{ck} = \mathbbm{1}\{\text{There is a college founded in $c$ that is available to $k$}\}$
  \end{wideitemize}
\end{frame}

\begin{frame}{The identification assumption}
\begin{wideitemize}
  \item Compare the gap in attainment between older cohorts and younger cohorts in counties that have a new college versus those that do not
  \item \textbf{Identifying assumption}: Conditional on controls, counties that gained a college and those that didn't would have experienced parallel trends in attainment across cohorts, absent the new college.
\end{wideitemize}
\end{frame}

\begin{frame}{Picking a control group}
\input{"/Users/cjwardius/Library/CloudStorage/OneDrive-UCSanDiego/demo of education/output/tables/county_classification_table.tex"}
\end{frame}




\begin{frame}{Testing parallel trends: Event study specification}
  To test for pre-trends and trace out dynamic effects, estimate:
  \begin{equation}
    y_{ick} = \alpha_c + \lambda_k + \sum_{j \neq -1} \beta_j \mathbbm{1}\{\text{Cohort } k \text{ born } j \text{ years relative to college founding in } c\} + \xi \bm{X}_{ick} + \epsilon_{ick}
  \end{equation}
  \begin{wideitemize}
    \item $j < 0$: Cohorts born \textit{before} college founding (test for pre-trends)
    \item $j \geq 0$: Cohorts born \textit{after} college founding (treatment effects)
    \item Omit $j = -1$ as reference category
    \item Null hypothesis: $\beta_j = 0$ for all $j < 0$ (no pre-trends)
  \end{wideitemize}
\end{frame}

\begin{frame}{Regional Control of Colleges}
    \input{"/Users/cjwardius/Library/CloudStorage/OneDrive-UCSanDiego/demo of education/output/tables/regional_control_colleges_table.tex"}
\end{frame}


\begin{frame}{What next?}
  \begin{wideitemize}
    \item Still cleaning my historical tables on enrollments
    \item Will have better data on program enrollments and associated analysis 
  \end{wideitemize}
\end{frame}

\begin{frame}{}
\centering cwardius@ucsd.edu
\end{frame}

\end{document}