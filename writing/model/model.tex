\documentclass[12pt]{article}

% Packages for math
\usepackage{amsmath}
\usepackage{amssymb}
\usepackage{amsthm}
\usepackage{mathtools}

% Packages for formatting
\usepackage[margin=1in]{geometry}
\usepackage{graphicx}
\usepackage{hyperref}

% Theorem environments
\newtheorem{theorem}{Theorem}
\newtheorem{lemma}{Lemma}
\newtheorem{proposition}{Proposition}
\newtheorem{definition}{Definition}
\newtheorem{assumption}{Assumption}

% Title information
\title{Simplified version of Hsiao 2024}
\author{Collin Wardius}
\date{\today}

\begin{document}

\maketitle


Individual $i$ with county of origin $j(i)$ and age group $k(i)$. There is some schooling shock $\epsilon$ and choose education $e$ (some school, completed high school, some college) to maximize utility.

\begin{align}
    u_{jk}(\epsilon) = \max_e\{\overbrace{\bar v_{jk}(e)}^{\text{labor utility}} - \underbrace{c_{jk}(e,\epsilon)}_{\text{education costs}}\}
\end{align}

\begin{align}
    \bar v_{jk}(e) = \mathbb{E}[v_{jk}(e,\epsilon)\mid e], \quad c_{jk}(e,\epsilon) = e\tau_{jk}^e\epsilon
\end{align}

In the second stage, the individual considers locations $l$. They take education $e$ as given and realize skill shocks $\epsilon = \{\epsilon_l\}$ across locations. They choose a destination to maximize labor utility: 

\begin{equation}
    v_{jk}(e,\epsilon) = \max_l\{v_{jkl}(e,\epsilon_l)\}    
\end{equation}

Labor utility is modeled in the following way:

\begin{equation}
    v_{jkl}(e,\epsilon) = \frac{a_lw_{jkl}(e,\epsilon_l)}{\tau_{jkl}^m},\quad w_{jkl}(e,\epsilon_l) = r_lh_{jkl}(e,\epsilon_l),\quad h_{jkl}(e,\epsilon_l) = e^\eta s_{jkl} \epsilon_l
\end{equation}

Wages depend on the wage rates $r_l$ per unit of human capital $h_{jkl}$. Human capital increases with education $e$ subject to decreasing marginal returns $\eta <1$. Human capital also increases in skill $s_{jkl}$ and skill shock $\epsilon_l$. The skill shocks follow the Frechet distribution: 

\begin{equation}
    F(\epsilon_1,\ldots, \epsilon_L) = \exp \{-\sum_l\epsilon_l^{-\theta}\}
\end{equation}
A high value of the Frechet parameter $\theta$ implies low skill dispersion. Migration costs are $\tau^m_{jkl}$ which are all the moving costs associated with moving away from home.

We can define how good a location is based on it's features other than idiosyncratic tastees by the following measure:

\begin{align}
    \tilde v_{jkl} = \frac{a_lr_ls_{jkl}}{\tau^m_{jkl}}, \quad MA_{jk} = \sum_l \tilde v_{jkl}^\theta
\end{align}

The reason for the weighting is as follows: when $\theta$ is high, there is relatively little skill dispersion, therefore the best locations will win out as locations with better fundamentals get more weight. The opposite happens when $\theta$ is small as there is more dispersion in skill and some people idiosyncratically like some locations worse or better.

\begin{equation}
    \bar e_{jk} = \mathbb{E}[e] = \left (\frac{\gamma\eta MA_{jk}^{\frac{1}{\theta}}}{\tau^e_{jk}} \right )^{\frac{1}{1-\eta}}\bar\epsilon
\end{equation}

\begin{align}
    \bar w_{jkl} = \mathbb{E} [w\mid \text{choose } l] = \left (\frac{\gamma \tau_{jkl}^m MA_{jk}^{\frac{1}{\theta}}}{a_l}\right ) \left (\frac{\bar e_{jk}}{\bar \epsilon} \right )^\eta \tilde e 
\end{align}

Then, we have the migration equation: 

\begin{align}
    \bar m_{jkl} = \mathbb{P}[\tilde v_{jkl}\epsilon_l\geq \tilde v_{jkl'}\epsilon_{l'} \text{ for all }l'] = \frac{\tilde v_{jkl}^\theta}{\sum_{l'}\tilde v^\theta_{jkl'} }
\end{align}


On the production side, we have national production given by $Y$, which sums across locations. Perfectly competitive firms produce with human capital $H_l$ subject to productivity $A_l$ and production elasticity $\kappa$. The production function determines demand for human capital.

\begin{equation}
    Y = \sum_l Y_l, \quad Y_l = A_lH_l^\kappa
\end{equation}

Since firms are perfect competitors, they will equate the wage rate to the marginal product of human capital:

\begin{equation}
    r_l = \kappa A_lH_l^{\kappa-1} = \frac{\partial Y_l}{\partial H_l}
\end{equation}

Total human capital sums across people as does wages

\begin{equation}
    H_l = \sum_{j,k} N_{j,k}\bar m_{jkl}\bar h_{jkl}, \quad W_l = \sum_{j,k} N_{j,k}\bar m_{jkl}\bar w_{jkl}
\end{equation}

$\bar m_{jkl}$ is the share of people from cohort $k$ that migrate to $l$ and have average human capital $h_{jkl}$ and average wages $\bar w_{jkl}$. Since $\bar w_{jkl} = r_l \bar h_{jkl}$, we have

\begin{equation}
    Y_l = \frac{r_lH_l}{\kappa} = \frac{W_l}{\kappa}
\end{equation}

Equilibrium is defined as the wage rates that clear the market in every labor market:

\begin{equation}
    H_l^D(r_l) = H_l^S(r) \quad \forall l
\end{equation}

Firms demand human capital. Production is only local, so for each location we have the following demand for human capital and demand can therefore be evaluated separately for every market:

\begin{equation}
    H_l^Dr_l = \left (\frac{\kappa A_l}{r_l} \right )^{\frac{1}{1-\kappa}}
\end{equation}

Supply is upward sloping and since people are able to migrate, we have 

\begin{align}
    H_l^S(r) = \frac{W_l(r)}{r(l)}
\end{align}


\end{document}
